\documentclass[12pt]{exam}
\usepackage{xifthen} %conditionals, to make writing different versions easier
\usepackage{latexsym}
\usepackage{amsmath}
\usepackage{amssymb}
\usepackage{amsthm}
\usepackage{mathtools}

\usepackage{textcomp}
%\usepackage{qtree}      %for truth trees
%\usepackage{fitch}      %for Fitch-style natural deduction proofs
\usepackage{centernot}

\usepackage{xcolor}
\newcommand{\red}[1]{{\color{red}#1}}
\newcommand{\green}[1]{{\color{teal}#1}}
\newcommand{\blue}[1]{{\color{blue}#1}}
\newcommand{\purple}[1]{{\color{purple}#1}}
\newcommand{\orange}[1]{{\color{orange}#1}}
\newcommand{\violet}[1]{{\color{violet}#1}}
\usepackage{ulem}

\newcommand{\redbullet}{\red{\bullet}}

\newcommand\FV[1]{\mbox{\rm FV}(#1)}
\newcommand{\boy}[1]{\mbox{\tt boy}(#1)}
\newcommand{\girl}[1]{\mbox{\tt girl}(#1)}
\newcommand{\knows}[2]{\mbox{\tt knows}(#1,#2)}
\newcommand{\expand}{{\tt expand}}
\newcommand{\mortal}{\tt mortal}
\newcommand{\human}{{\tt human}}
\newcommand{\spartan}{{\tt spartan}}
\newcommand{\brave}{{\tt brave}}
\newcommand{\greek}{{\tt greek}}
\newcommand{\perfect}{{\tt perfect}}
\newcommand{\wise}{{\tt wise}}
\newcommand{\philosopher}{{\tt philosopher}}
\newcommand{\mathematician}{{\tt mathematician}}
\newcommand{\admires}{{\tt admires}}
\newcommand{\critic}{{\tt critic}}
\newcommand{\contradicts}{{\tt contradicts}}
\newcommand{\rectangle}{{\tt rectangle}}
\newcommand{\figre}{{\tt figure}}
\newcommand{\draw}{{\tt draws}}
\newcommand{\painting}{{\tt painting}}
\newcommand{\constant}[1]{{\bf #1}}

\newcommand{\takes}[2]{\mathtt{takes}(#1,#2)}
\newcommand{\teaches}[2]{\mathtt{teaches}(#1,#2)}
\newcommand{\enroll}[2]{\mathtt{enroll}(#1,#2)}

\newcommand{\person}[1]{\mathtt{person}(#1)}
\newcommand{\student}[1]{\mathtt{student}(#1)}
\newcommand{\major}[2]{\mathtt{hasMajor}(#1,#2)}
\newcommand{\professor}[1]{\mathtt{professor}(#1)}
\newcommand{\course}[1]{\mathtt{course}(#1)}

\newcommand*{\defeq}{\mathrel{\vcenter{\baselineskip0.5ex \lineskiplimit0pt
                     \hbox{\scriptsize.}\hbox{\scriptsize.}}}%
                     =}

\newcommand{\wicked}{{\tt wicked}}
\newcommand{\UI}{{\tt UI}}
\newcommand{\EI}{{\tt EI}}
\newcommand{\UG}{{\tt UG}}
\newcommand{\EG}{{\tt EG}}

\newcommand{\slide}{\qquad\qquad\qquad\qquad\qquad}

\newcommand{\argument}[3]
{\begin{center}
 \begin{tabular}{l}
 #1 \\
 #2 \\ \hline
 #3 
 \end{tabular}
 \end{center}
}

\newcommand{\argumentthree}[4]
{\begin{center}
 \begin{tabular}{l}
 #1 \\
 #2 \\ 
 #3 \\ \hline
 #4 
 \end{tabular}
 \end{center}
}

\newcommand{\argumentone}[2]
{\begin{center}
 \begin{tabular}{l}
 #1 \\ \hline
 #2 
 \end{tabular}
 \end{center}
}

\newcommand{\li}[3]
{\begin{center}
 \begin{tabular}{l}
 $#1$, \\
 $#2$ \\ 
 $\Limplies$ \\
 $#3$
 \end{tabular}
 \end{center}
}


\newcommand{\lithree}[4]
{\begin{center}
 \begin{tabular}{l}
 $#1$, \\
 $#2$ \\ 
 $#3$ \\
 $\Limplies$ \\
 $#4$
 \end{tabular}
 \end{center}
}



\newcommand{\lione}[2]
{\begin{center}
 \begin{tabular}{l}
 $#1$ \\
 $\Limplies$ \\
 $#2$
 \end{tabular}
 \end{center}
}

\newcommand{\bs}{\textbackslash}
\newcommand{\st}{\mathrel{\vert}}
\newcommand{\Or}{\mathbin{\vee}}
\newcommand{\relor}{\mathrel{\vee}}
\renewcommand{\And}{\mathbin{\wedge}}
\newcommand{\reland}{\mathrel{\wedge}}
\newcommand{\Implies}{\mathbin{\rightarrow}}
\newcommand{\Lequiv}{\mathbin{\equiv}}
\newcommand{\Limplies}{\mathbin{\vDash}}
\newcommand{\Iff}{\mathbin{\leftrightarrow}}
\newcommand{\Not}{\mathop{\neg}}
\newcommand{\Nor}{\mathbin{\downarrow}}
\newcommand{\Z}{\mathbb{Z}}
\newcommand{\N}{\mathbb{N}}
\newcommand{\R}{\mathbb{R}}
\newcommand{\Q}{\mathbb{Q}}
\newcommand{\C}{\mathbb{C}}
\newcommand{\nothing}{\varnothing}
\newcommand{\power}{\mathop{\mathcal{P}}}
\newcommand{\Univ}{\mathcal{U}}
\newcommand{\isect}{\mathbin{\cap}}
\newcommand{\union}{\mathbin{\cup}}
\newcommand{\comp}[1]{\overline{#1}}
\newcommand{\blank}{\underline{\hspace{1em}}}
\newcommand{\newterm}[1]{\textbf{#1}}
\newcommand{\True}{\mathrm{T}}
\newcommand{\False}{\mathrm{F}}
\newcommand{\T}{\True}
\newcommand{\F}{\False}
\newcommand{\<}{\langle}
\renewcommand{\>}{\rangle}
\newcommand{\attn}[1]{\textbf{#1}}
\newcommand{\note}[1]{{\footnotesize \textbf{Note:} \textit{#1}}}
\newcommand{\openbranch}{$\circ$}
\newcommand{\closebranch}{$\times$}

\newcommand{\cM}{\mathcal{M}}
\newcommand{\cI}{\mathcal{I}}
\newcommand{\cJ}{\mathcal{J}}
\newcommand{\cN}{\mathcal{N}}

\theoremstyle{definition}   %bold header, roman body
 \newtheorem{example}{Example}
 \newtheorem*{question}{Question}
 \newtheorem*{answer}{Answer}
 \newtheorem*{claim}{Claim}



\begin{document}
\pagestyle{headandfoot}
\firstpageheader{\bf DSCI-D 321 Assignment  4 \\
                      Instructor: Dirk Van Gucht
                }{}{Assigned Thursday, February 16, 2023\\
                    Due Wednesday 1, 2023 
				}



\vfill
\noindent
{\bf Discussed assignment with}:
\noindent
\vskip 3cm
In this assignment, you will be asked to write sentences and queries in Predicate Logic, SQL and
Python.   You will also be required to test your SQL and Python code on some data.

You will need to use PostgreSQL to implement your SQL code.
And, you will need a Python programming language environment to implement your Python code.

Along with this assignment, there are two files 
\begin{center}
`{\tt assignment4Script.sql}' and `{\tt assignment4Script.py}'
\end{center}
that contain
the data that you will need to use for this assignment.

You will need to submit 3 files:
\begin{enumerate}
\item A file with name `{\tt assignment4.pdf}' 
that contains the solutions for problems indicated with a red bullet $\redbullet$.
This file should be generated using Latex;
\item A file with name `{\tt assignment4.sql}' that contains
the solutions for the problems requiring SQL code; and 
\item A file with name `{\tt assignment4.py}' that contains
solutions for the problems requiring Python code.
\end{enumerate}

\bigskip
Each problem or subproblem is worth 10 points.

\newpage
%\newpage
%\paragraph {Preliminaries}
%
%Consider three domains `$\mathbf{P}$' (the domain of persons), `$\mathbf{C}$' (the domain of courses),
%and `$\mathbf{M}$' (the domain of majors).
%
%\begin{itemize}
%\item Variables that range over `$\mathbf{P}$' should be named $p$, $p_1$, $p_2$, etc.
%\item Variables that range over `$\mathbf{C}$' should be named $c$, $c_1$, $c_2$, etc.
%\item Variables that range over `$\mathbf{M}$' should be named $m$, $m_1$, $m_2$, etc.
%\end{itemize}
%
%
%Consider the following unary predicates
%\begin{center}
%\begin{tabular}{lcl}
%$\student{p}$ & $\defeq$ & ``person $p$ is a student"\\
%$\professor{p}$ & $\defeq$ & ``person $p$ is a professor"\\
%\end{tabular}
%\end{center}
%and consider the following binary predicates
%\begin{center}
%\begin{tabular}{lcl}
%$\major{p}{m}$& &  ``person $p$ has major $m$"\\
%$\enroll{p}{c}$ &&  ``person $p$ is enrolled (i.e., takes) course $c$" \\
%$\teaches{p}{c}$ &&  ``person $p$ takes course $c$" \\
%$\knows{p_1}{p_2}$ &&  ``person $p_1$ knows person $p_2$" \\
%\end{tabular}
%\end{center}
%
%\newpage

%\paragraph{Expressing Predicate Logic Sentences in SQL and Python}
\medskip
\begin{questions}
  \question $\redbullet$\ The domain of the following predicates is the set of all plants.
  Consider the following predicates:
  
  \begin{center}
  \begin{tabular}{lcl}
  $P(x)$ & $=$ & ``$x$ is poisonous." \\
  $Q(x)$ & $=$ & ``Jeff has eaten $x$." \\
  \end{tabular}
  \end{center}

Express the following sentences into Predicate Logic.
\begin{parts}
\item ``Some plants are not poisonous." \\
$\exists x (\neg P(x)) $
\item ``Jeff has eaten a poisonous plant as well as a non-poisonous plant." \\
$\exists x \exists y (P(x) \land \neg P(y) \land Q(x) \land Q(y)) $
\item ``There are some non-poisonous plants that Jeff has not eaten." \\
$\exists x (\neg P(x) \land \neg Q(x))$
\end{parts}


  \question $\redbullet$\ The domain of the following predicates is the set of all books.
  Consider the following predicates:
  
  \begin{center}
  \begin{tabular}{lcl}
  $H(x)$ & $=$ & ``$x$ is heavy." \\
  $C(x)$ & $=$ & ``$x$ is confusing." \\
  \end{tabular}
  \end{center}

Express the following Predicate Logic sentences into ordinary English.
\begin{parts}
\item $\forall x\, (H(x) \Implies C(x))$. \\
"If a book is heavy, it is confusing."
\item $\Not \exists x (H(x) \And \Not C(x))$. \\
"There is not a heavy book that is not confusing."
\item $\forall x\, (H(x) \Implies (\exists x\, C(x))$. \\
"If all the books are heavy, there is a book that is confusing."
\end{parts}

\question\label{predicateLogic} 
$\redbullet$\ 
Consider the predicates $P(x)$, $M(x)$, and
$A(x,y)$ in a domain of people.  The predicate $P(x)$
states of a person that he or she is a philosopher, the predicate $M(x)$ states
of a person that he or she is a mathematician, and the predicate $A(x,y)$ states
that $x$ is an admirer of $y$.  Write a sentence in Predicate Logic for
the following natural languages sentences.

\begin{parts}
\part Each philosopher admires a mathematician.\\
$\forall x \exists y (P(x) \to (M(y) \land A(x, y)))$
\part Not everyone admires a philosopher who is also a mathematician. \\
$\neg \forall x(P(x) \land M(x) \to \exists y(A(x, y) \land P(y) \land M(y)))$
\part Some person admires all philosophers but only some mathematicians. \\
$\exists z \forall x ((P(x) \land \neg M(x)) \to A(z, x)) \land (\exists y (M(y) \land A(z, y)))$
\part Some person admires a philosopher or a mathematician, but not both. \\
$\exists x \forall y \forall z ((P(y) \land M(z) \land y \neq z) \to ((A(x,y) \land \neg A(x,z)) \lor (\neg A(x,y) \land A(x,z))))$
\end{parts}
%\newpage

\question $\redbullet$\ Reconsider the predicates of question~\ref{predicateLogic}.

Express the following queries in the Predicate Logic.

\begin{parts}
\part Find each person who admires a philosopher. \\
$\exists x \exists y (P(y) \land A(x,y))$
\part Find each philosopher who admires a mathematician who admires all philsophers.
$\exists x \exists y ((P(y) \land \forall z (P(z) \rightarrow A(y,z))) \land M(x) \land \forall z (P(z) \rightarrow (A(x,z) \land A(y,x))))$
\part Find each pair of persons $(x,y)$ such that if $x$ admires a philosopher then $y$ also admires that philosopher.\\
$\forall x \forall y ((\exists z(P(z) \land A(x,z))) \land (A(x,y) \rightarrow (\exists z(P(z) \land A(y,z))))$
\end{parts}

\newpage
\paragraph{Expressing sentences and queries in SQL and Python} \ 


\noindent
Consider three domains `$\mathbf{P}$' (the domain of persons), `$\mathbf{C}$' (the domain of courses),
and `$\mathbf{M}$' (the domain of majors).

\begin{itemize}
\item Variables that range over `$\mathbf{P}$' should be named $p$, $p_1$, $p_2$, etc.
\item Variables that range over `$\mathbf{C}$' should be named $c$, $c_1$, $c_2$, etc.
\item Variables that range over `$\mathbf{M}$' should be named $m$, $m_1$, $m_2$, etc.
\end{itemize}


Consider the unary predicates
\begin{center}
\begin{tabular}{lcl}
$\student{p}$ & $\defeq$ & ``person $p$ is a student"\\
$\professor{p}$ & $\defeq$ & ``person $p$ is a professor"\\
\end{tabular}
\end{center}
and the binary predicates
\begin{center}
\begin{tabular}{lcl}
$\major{p}{m}$& &  ``person $p$ has major $m$"\\
$\enroll{p}{c}$ &&  ``person $p$ is enrolled in (i.e., takes) course $c$" \\
$\teaches{p}{c}$ &&  ``person $p$ takes course $c$" \\
$\knows{p_1}{p_2}$ &&  ``person $p_1$ knows person $p_2$" \\
\end{tabular}
\end{center}




%For each of the following Predicate Logic sentences, 
%write an SQL statement and a Python statement that expresses this sentence.
%Then evaluate these statements in PostgreSQL and Python.
%As data you should used the data in the `{\tt assignment4.sql}' and
%`{\tt assignment4.py}' files.    

%%\begin{questions}


\question




Consider the sentence ``\emph{Some student knows a professor who teaches the `Databases' course.}''
This sentence can be expressed in Predicate Logic as follows:
\begin{multline*}
\exists p_1(\student{p_1}\,\And\, \exists p_2( \professor{p_2} \And \knows{p_1}{p_2}\And 
\teaches{p_2}{\mathbf{Databases}})).
\end{multline*}
\begin{parts}
\item  Express this Predicate Logic sentence as an SQL statement.   Place your answer in the `{\tt assignment4.sql}' file.
\item  Express this Predicate Logic sentence as a Python statement.   Place your answer in the `{\tt assignment4.py}' file.
\end{parts}

\question 
Consider the sentence ``\emph{Each course taught by professor `Anna' is taken by at least two students}."
This sentence can be expressed in Predicate Logic as follows:
\begin{multline*}
\forall c((\teaches{c}{\mathbf{Anna}}\And \professor{\mathbf{Anna}} \Implies \\
\exists p_1\exists p_2(p_1\neq p_2\And \student{p_1}\And \student{p_2}\And \enroll{p_1}{c} \And \enroll{p_2}{c})).
\end{multline*}
\begin{parts}
\part  Express this Predicate Logic sentence as an SQL statement.   Place your answer in the `{\tt assignment4.sql}' file.
\part  Express this Predicate Logic sentence as a Python statement.   Place your answer in the `{\tt assignment4.py}' file.
\part Rewrite the Predicate Logic sentence above into a logically equivalent Predicate Logic sentence in which only existential quantifiers ($\exists$) appear.
In other words, rewrite all the universal quantifiers.
(Hint: use Replacement Laws of Predicate Logic for quantifiers and Predicate Logic sentences.)
\part  Express the resulting Predicate Logic sentence as an SQL statement.   Place your answer in the `{\tt assignment4.sql}' file.
\part  Express the resulting Predicate Logic sentence as a Python statement.   Place your answer in the `{\tt assignment4.py}' file.
\end{parts}



%\newpage

%\paragraph{Expressing Predicate Logic Queries in SQL and Python}
%\vskip 1cm
%\noindent
%For each of the following Predicate Logic queries, you will be asked to 
%write an SQL query and a Python query (i.e., comprehension) that expresses this query.
%Then evaluate this query in PostgreSQL and Python.
%As data you should used the data in the `{\tt assignment4.sql}' and
%`{\tt assignment4.py}' files.

\question
Consider the query ``\emph{Find the majors of students who are enrolled in the course `Algorithms'}."
\begin{parts}
\item  Express this query as an SQL query.   Place your answer in the `{\tt assignment4.sql}' file.
\item  Express this query as a Python query (comprehension).   Place your answer in the `{\tt assignment4.py}' file.
\end{parts}

\question
Consider the query ``\emph{Find each student who knows a student who takes a taught by Professor `Emma' or a course taught by professor `Arif' or a
course taught by professor `Anna'.}"

\begin{parts}
\item  Express this query as an SQL query.   Place your answer in the `{\tt assignment4.sql}' file.
\item  Express this query as a Python query (comprehension).   Place your answer in the `{\tt assignment4.py}' file.
\end{parts}

\question
Consider the query ``\emph{Find each pair of different students who both know a same professor who teaches the course `Databases'}."

\begin{parts}
\item  Express this query as an SQL query.   Place your answer in the `{\tt assignment4.sql}' file.
\item  Express this query as a Python query (comprehension).   Place your answer in the `{\tt assignment4.py}' file.
\end{parts}

\question
Consider the query ``\emph{Find each professor who only teaches courses taken by all students who major in `DataScience'.}''

\begin{parts}
\item  Express this query as an SQL query.   Place your answer in the `{\tt assignment4.sql}' file.
\item  Express this query as a Python query (comprehension).   Place your answer in the `{\tt assignment4.py}' file.
\end{parts}

\question
Consider the query ``\emph{Find each professor who does not know any student who majors in both `DataScience' and in `Chemistry'}."
\begin{parts}
\item  Express this query as an SQL query.   Place your answer in the `{\tt assignment4.sql}' file.
\item  Express this query as a Python query (comprehension).   Place your answer in the `{\tt assignment4.py}' file.
\end{parts}

\question\label{sentenceThree} Consider the sentence ``\emph{Find each pair of different students who have a common major and who take none of the courses taught by professor `Pedro'}."

\begin{parts}
\item  Express the resulting sentence as an SQL statement.   Place your answer in the `{\tt assignment4.sql}' file.
\item  Express the resulting sentence as a Python statement.   Place your answer in the `{\tt assignment4.py}' file.
\end{parts}

%\newpage
%\paragraph{Predicate Logic semantics}
%
%\medskip
%\noindent
%The following questions rely on an understanding of the semantics of Predicate Logic.
%You should answer these question in the space provided.
%
%\question
%Prove, without use the Replacement Laws for quantifiers, that 
%\[\Not \forall x\, (P(x)\Implies Q(x)) \Lequiv \exists x (P(x)\And \Not Q(x)).\]
%I.e., use the method of expanding universal and existential quantifiers to establish this logical equivalence.
%\vskip 6 cm
%\newpage
%\question
%Use replacement laws for Predicate Logic to establish the following logical equivalence:
%\[\forall x\left(((P(x) \Implies Q(y))\And \Not Q(y))\Implies \Not P(x)\right)\Lequiv \mathbf{T}.\]
\end{questions}

\end{document}

