% Options for packages loaded elsewhere
\PassOptionsToPackage{unicode}{hyperref}
\PassOptionsToPackage{hyphens}{url}
%
\documentclass[
]{article}
\usepackage{amsmath,amssymb}
\usepackage{lmodern}
\usepackage{iftex}
\ifPDFTeX
  \usepackage[T1]{fontenc}
  \usepackage[utf8]{inputenc}
  \usepackage{textcomp} % provide euro and other symbols
\else % if luatex or xetex
  \usepackage{unicode-math}
  \defaultfontfeatures{Scale=MatchLowercase}
  \defaultfontfeatures[\rmfamily]{Ligatures=TeX,Scale=1}
\fi
% Use upquote if available, for straight quotes in verbatim environments
\IfFileExists{upquote.sty}{\usepackage{upquote}}{}
\IfFileExists{microtype.sty}{% use microtype if available
  \usepackage[]{microtype}
  \UseMicrotypeSet[protrusion]{basicmath} % disable protrusion for tt fonts
}{}
\makeatletter
\@ifundefined{KOMAClassName}{% if non-KOMA class
  \IfFileExists{parskip.sty}{%
    \usepackage{parskip}
  }{% else
    \setlength{\parindent}{0pt}
    \setlength{\parskip}{6pt plus 2pt minus 1pt}}
}{% if KOMA class
  \KOMAoptions{parskip=half}}
\makeatother
\usepackage{xcolor}
\usepackage[margin=1in]{geometry}
\usepackage{color}
\usepackage{fancyvrb}
\newcommand{\VerbBar}{|}
\newcommand{\VERB}{\Verb[commandchars=\\\{\}]}
\DefineVerbatimEnvironment{Highlighting}{Verbatim}{commandchars=\\\{\}}
% Add ',fontsize=\small' for more characters per line
\usepackage{framed}
\definecolor{shadecolor}{RGB}{248,248,248}
\newenvironment{Shaded}{\begin{snugshade}}{\end{snugshade}}
\newcommand{\AlertTok}[1]{\textcolor[rgb]{0.94,0.16,0.16}{#1}}
\newcommand{\AnnotationTok}[1]{\textcolor[rgb]{0.56,0.35,0.01}{\textbf{\textit{#1}}}}
\newcommand{\AttributeTok}[1]{\textcolor[rgb]{0.77,0.63,0.00}{#1}}
\newcommand{\BaseNTok}[1]{\textcolor[rgb]{0.00,0.00,0.81}{#1}}
\newcommand{\BuiltInTok}[1]{#1}
\newcommand{\CharTok}[1]{\textcolor[rgb]{0.31,0.60,0.02}{#1}}
\newcommand{\CommentTok}[1]{\textcolor[rgb]{0.56,0.35,0.01}{\textit{#1}}}
\newcommand{\CommentVarTok}[1]{\textcolor[rgb]{0.56,0.35,0.01}{\textbf{\textit{#1}}}}
\newcommand{\ConstantTok}[1]{\textcolor[rgb]{0.00,0.00,0.00}{#1}}
\newcommand{\ControlFlowTok}[1]{\textcolor[rgb]{0.13,0.29,0.53}{\textbf{#1}}}
\newcommand{\DataTypeTok}[1]{\textcolor[rgb]{0.13,0.29,0.53}{#1}}
\newcommand{\DecValTok}[1]{\textcolor[rgb]{0.00,0.00,0.81}{#1}}
\newcommand{\DocumentationTok}[1]{\textcolor[rgb]{0.56,0.35,0.01}{\textbf{\textit{#1}}}}
\newcommand{\ErrorTok}[1]{\textcolor[rgb]{0.64,0.00,0.00}{\textbf{#1}}}
\newcommand{\ExtensionTok}[1]{#1}
\newcommand{\FloatTok}[1]{\textcolor[rgb]{0.00,0.00,0.81}{#1}}
\newcommand{\FunctionTok}[1]{\textcolor[rgb]{0.00,0.00,0.00}{#1}}
\newcommand{\ImportTok}[1]{#1}
\newcommand{\InformationTok}[1]{\textcolor[rgb]{0.56,0.35,0.01}{\textbf{\textit{#1}}}}
\newcommand{\KeywordTok}[1]{\textcolor[rgb]{0.13,0.29,0.53}{\textbf{#1}}}
\newcommand{\NormalTok}[1]{#1}
\newcommand{\OperatorTok}[1]{\textcolor[rgb]{0.81,0.36,0.00}{\textbf{#1}}}
\newcommand{\OtherTok}[1]{\textcolor[rgb]{0.56,0.35,0.01}{#1}}
\newcommand{\PreprocessorTok}[1]{\textcolor[rgb]{0.56,0.35,0.01}{\textit{#1}}}
\newcommand{\RegionMarkerTok}[1]{#1}
\newcommand{\SpecialCharTok}[1]{\textcolor[rgb]{0.00,0.00,0.00}{#1}}
\newcommand{\SpecialStringTok}[1]{\textcolor[rgb]{0.31,0.60,0.02}{#1}}
\newcommand{\StringTok}[1]{\textcolor[rgb]{0.31,0.60,0.02}{#1}}
\newcommand{\VariableTok}[1]{\textcolor[rgb]{0.00,0.00,0.00}{#1}}
\newcommand{\VerbatimStringTok}[1]{\textcolor[rgb]{0.31,0.60,0.02}{#1}}
\newcommand{\WarningTok}[1]{\textcolor[rgb]{0.56,0.35,0.01}{\textbf{\textit{#1}}}}
\usepackage{graphicx}
\makeatletter
\def\maxwidth{\ifdim\Gin@nat@width>\linewidth\linewidth\else\Gin@nat@width\fi}
\def\maxheight{\ifdim\Gin@nat@height>\textheight\textheight\else\Gin@nat@height\fi}
\makeatother
% Scale images if necessary, so that they will not overflow the page
% margins by default, and it is still possible to overwrite the defaults
% using explicit options in \includegraphics[width, height, ...]{}
\setkeys{Gin}{width=\maxwidth,height=\maxheight,keepaspectratio}
% Set default figure placement to htbp
\makeatletter
\def\fps@figure{htbp}
\makeatother
\setlength{\emergencystretch}{3em} % prevent overfull lines
\providecommand{\tightlist}{%
  \setlength{\itemsep}{0pt}\setlength{\parskip}{0pt}}
\setcounter{secnumdepth}{-\maxdimen} % remove section numbering
\ifLuaTeX
  \usepackage{selnolig}  % disable illegal ligatures
\fi
\IfFileExists{bookmark.sty}{\usepackage{bookmark}}{\usepackage{hyperref}}
\IfFileExists{xurl.sty}{\usepackage{xurl}}{} % add URL line breaks if available
\urlstyle{same} % disable monospaced font for URLs
\hypersetup{
  pdftitle={homework5\_b365},
  pdfauthor={Morgan Fissel},
  hidelinks,
  pdfcreator={LaTeX via pandoc}}

\title{homework5\_b365}
\author{Morgan Fissel}
\date{2023-03-22}

\begin{document}
\maketitle

\#Problem 1 - a) Reasoning from this plot, choose a variable and a split
point for the variable so that the resulting two regions have one that
contains only Setosa flowers, while the other mixes the other two
classes.

\begin{Shaded}
\begin{Highlighting}[]
\CommentTok{\# We consider Fisher\textquotesingle{}s famous iris data set and visualize data both as scatterplot and pairs plot}
\CommentTok{\# In classification we try predict the "label" or "class" of an observation from the variables we measure.}


\FunctionTok{data}\NormalTok{(iris);         }\CommentTok{\# include the famous iris data}
\NormalTok{n }\OtherTok{=} \FunctionTok{nrow}\NormalTok{(iris);     }\CommentTok{\# n is number of observations (will usually use "n" for this)}
\NormalTok{type }\OtherTok{=} \FunctionTok{rep}\NormalTok{(}\DecValTok{0}\NormalTok{,n);}
\NormalTok{color }\OtherTok{=} \FunctionTok{rep}\NormalTok{(}\DecValTok{0}\NormalTok{,n);}

\NormalTok{type[iris[,}\DecValTok{5}\NormalTok{] }\SpecialCharTok{==} \StringTok{"setosa"}\NormalTok{] }\OtherTok{=} \StringTok{"s"}\NormalTok{;      }\CommentTok{\# class is 5th column.  }
\NormalTok{type[iris[,}\DecValTok{5}\NormalTok{] }\SpecialCharTok{==} \StringTok{"versicolor"}\NormalTok{] }\OtherTok{=} \StringTok{"c"}\NormalTok{;}
\NormalTok{type[iris[,}\DecValTok{5}\NormalTok{] }\SpecialCharTok{==} \StringTok{"virginica"}\NormalTok{] }\OtherTok{=} \StringTok{"v"}\NormalTok{;}

\NormalTok{color[iris[,}\DecValTok{5}\NormalTok{] }\SpecialCharTok{==} \StringTok{"setosa"}\NormalTok{] }\OtherTok{=} \DecValTok{1}\NormalTok{;}
\NormalTok{color[iris[,}\DecValTok{5}\NormalTok{] }\SpecialCharTok{==} \StringTok{"versicolor"}\NormalTok{] }\OtherTok{=} \DecValTok{2}\NormalTok{;}
\NormalTok{color[iris[,}\DecValTok{5}\NormalTok{] }\SpecialCharTok{==} \StringTok{"virginica"}\NormalTok{] }\OtherTok{=} \DecValTok{3}\NormalTok{;}


\CommentTok{\# type is now a vector of "s" or "c" or "v" for 3 types}


\CommentTok{\#   here is a scatterplot}

\CommentTok{\#plot(iris[,1],iris[,2],pch=type)  \# scatterplot}
\FunctionTok{plot}\NormalTok{(iris[,}\DecValTok{3}\NormalTok{],iris[,}\DecValTok{2}\NormalTok{],}\AttributeTok{pch=}\NormalTok{type,}\AttributeTok{col=}\NormalTok{color)  }\CommentTok{\# scatterplot}
\end{Highlighting}
\end{Shaded}

\includegraphics{homework5_b365_files/figure-latex/unnamed-chunk-1-1.pdf}

\begin{Shaded}
\begin{Highlighting}[]
\CommentTok{\# alternaltively could use}
\CommentTok{\# plot(iris[,"Sepal.Length"],iris[,"Sepal.Width"],pch=type)}

\CommentTok{\# here is pairs plot}


\CommentTok{\#pairs(iris[,1:4],pch=type,col=color,cex=2)   \# pairs plot using all four variables}
\FunctionTok{print}\NormalTok{(}\StringTok{\textquotesingle{}Split point at x = 2.5 petal length\textquotesingle{}}\NormalTok{)}
\end{Highlighting}
\end{Shaded}

\begin{verbatim}
## [1] "Split point at x = 2.5 petal length"
\end{verbatim}

\begin{itemize}
\item
  \begin{enumerate}
  \def\labelenumi{\alph{enumi})}
  \setcounter{enumi}{1}
  \tightlist
  \item
    For the ``mixed'' region resulting from the first part, choose a
    variable and split point that separates the Versicolor and Virginica
    flowers as well as possible.
  \end{enumerate}
\end{itemize}

\begin{Shaded}
\begin{Highlighting}[]
\FunctionTok{print}\NormalTok{(}\StringTok{\textquotesingle{}Split point at x = 5 petal length would seperate the Versicolor and Virginica flowers well.\textquotesingle{}}\NormalTok{)}
\end{Highlighting}
\end{Shaded}

\begin{verbatim}
## [1] "Split point at x = 5 petal length would seperate the Versicolor and Virginica flowers well."
\end{verbatim}

\begin{itemize}
\item
  \begin{enumerate}
  \def\labelenumi{\alph{enumi})}
  \setcounter{enumi}{2}
  \tightlist
  \item
    Sketch the resulting three regions over the scatterplot of the two
    relevant variables, clearly labeling each region with the resulting
    class.
  \end{enumerate}
\end{itemize}

\begin{Shaded}
\begin{Highlighting}[]
\FunctionTok{plot}\NormalTok{(iris[,}\DecValTok{3}\NormalTok{],iris[,}\DecValTok{2}\NormalTok{],}\AttributeTok{pch=}\NormalTok{type,}\AttributeTok{col=}\NormalTok{color)}
\FunctionTok{abline}\NormalTok{(}\AttributeTok{v =} \FunctionTok{c}\NormalTok{(}\FloatTok{2.5}\NormalTok{, }\DecValTok{5}\NormalTok{), }\AttributeTok{lty =} \DecValTok{2}\NormalTok{, }\AttributeTok{col =} \FunctionTok{c}\NormalTok{(}\StringTok{\textquotesingle{}blue\textquotesingle{}}\NormalTok{, }\StringTok{\textquotesingle{}red\textquotesingle{}}\NormalTok{))}
\FunctionTok{text}\NormalTok{(}\FloatTok{2.2}\NormalTok{, }\FloatTok{3.5}\NormalTok{, }\StringTok{"setosa"}\NormalTok{, }\AttributeTok{col =} \StringTok{"black"}\NormalTok{)}
\FunctionTok{text}\NormalTok{(}\FloatTok{3.8}\NormalTok{, }\FloatTok{3.5}\NormalTok{, }\StringTok{"versicolor"}\NormalTok{, }\AttributeTok{col =} \StringTok{"red"}\NormalTok{)}
\FunctionTok{text}\NormalTok{(}\DecValTok{6}\NormalTok{, }\FloatTok{3.5}\NormalTok{, }\StringTok{"virginica"}\NormalTok{, }\AttributeTok{col =} \StringTok{"green"}\NormalTok{)}
\end{Highlighting}
\end{Shaded}

\includegraphics{homework5_b365_files/figure-latex/unnamed-chunk-3-1.pdf}

\#Problem 2 - On PDF separate

\#Problem 3 - a) Write a recursive function in R that takes as input the
number of a node and returns the optimal risk associated with that node,
with a split penalty of α = .03. When you run your function with input 1
(the root node) it should return the optimal risk for the entire tree.

\begin{Shaded}
\begin{Highlighting}[]
\NormalTok{X }\OtherTok{=} \FunctionTok{matrix}\NormalTok{(}\FunctionTok{scan}\NormalTok{(}\StringTok{"tree\_data.dat"}\NormalTok{),}\AttributeTok{byrow=}\NormalTok{T,}\AttributeTok{ncol =} \DecValTok{3}\NormalTok{)}
\NormalTok{n }\OtherTok{=}\NormalTok{ X[}\DecValTok{1}\NormalTok{,}\DecValTok{1}\NormalTok{] }\SpecialCharTok{+}\NormalTok{ X[}\DecValTok{1}\NormalTok{,}\DecValTok{2}\NormalTok{]}

\NormalTok{optimal\_risk }\OtherTok{=} \ControlFlowTok{function}\NormalTok{(node\_num)\{}
  \ControlFlowTok{if}\NormalTok{ (X[node\_num,}\DecValTok{3}\NormalTok{]}\SpecialCharTok{==}\DecValTok{1}\NormalTok{)\{}
\NormalTok{    prune\_cost }\OtherTok{=} \FunctionTok{min}\NormalTok{(X[node\_num,}\DecValTok{1}\NormalTok{],X[node\_num,}\DecValTok{2}\NormalTok{])}
\NormalTok{    prune\_cost }\OtherTok{=}\NormalTok{ prune\_cost}\SpecialCharTok{/}\NormalTok{n}
    \FunctionTok{return}\NormalTok{(prune\_cost)}
\NormalTok{    \}}\ControlFlowTok{else}\NormalTok{\{}
\NormalTok{      prune\_cost }\OtherTok{=} \FunctionTok{min}\NormalTok{(X[node\_num,}\DecValTok{1}\NormalTok{],X[node\_num,}\DecValTok{2}\NormalTok{])}
\NormalTok{      prune\_cost }\OtherTok{=}\NormalTok{ prune\_cost}\SpecialCharTok{/}\NormalTok{n}
\NormalTok{      split\_cost }\OtherTok{=}\NormalTok{ (.}\DecValTok{03} \SpecialCharTok{+} \FunctionTok{optimal\_risk}\NormalTok{(}\DecValTok{2}\SpecialCharTok{*}\NormalTok{node\_num) }\SpecialCharTok{+} \FunctionTok{optimal\_risk}\NormalTok{(}\DecValTok{2}\SpecialCharTok{*}\NormalTok{node\_num}\SpecialCharTok{+}\DecValTok{1}\NormalTok{))}
      \FunctionTok{return}\NormalTok{(}\FunctionTok{min}\NormalTok{(split\_cost,prune\_cost))}
\NormalTok{    \}}
\NormalTok{\}}
\FunctionTok{optimal\_risk}\NormalTok{(}\DecValTok{1}\NormalTok{)}
\end{Highlighting}
\end{Shaded}

\begin{verbatim}
## [1] 0.32
\end{verbatim}

\begin{itemize}
\item
  \begin{enumerate}
  \def\labelenumi{\alph{enumi})}
  \setcounter{enumi}{1}
  \tightlist
  \item
    Let Tα=.03 denote the associated optimal tree, as computed in the
    previous problem. Construct this tree, explicitly giving the
    classifications associated with each terminal tree node.
  \end{enumerate}
\end{itemize}

\begin{Shaded}
\begin{Highlighting}[]
\NormalTok{X[}\DecValTok{4}\NormalTok{,}\DecValTok{3}\NormalTok{] }\OtherTok{=} \DecValTok{1}
\NormalTok{Y }\OtherTok{=}\NormalTok{ X[}\DecValTok{1}\SpecialCharTok{:}\DecValTok{7}\NormalTok{,]}
\FunctionTok{print}\NormalTok{(Y)}
\end{Highlighting}
\end{Shaded}

\begin{verbatim}
##      [,1] [,2] [,3]
## [1,]   40   60    0
## [2,]   22   35    0
## [3,]   18   25    0
## [4,]   15    5    1
## [5,]    7   30    1
## [6,]   12    5    1
## [7,]    6   20    1
\end{verbatim}

Terminal nodes are: 4,5,6,7 with respective classifications, +,-,+,-

\begin{itemize}
\item
  \begin{enumerate}
  \def\labelenumi{\alph{enumi})}
  \setcounter{enumi}{2}
  \tightlist
  \item
    Increase alpha to a value of 0.08 to prune at node 3. And alpha of
    0.02 should decrease enough to include more splits.
  \end{enumerate}
\end{itemize}

\#Problem 4 - a) In the ``rel error'' column we get a value of 0. for
the 23rd row. Explain what this number means. A relative error of 0 on
the largest number of splits indicates that the model is fitting the
training data too closely and is not generalizing well to new data.
Likely overfitting.

\begin{itemize}
\item
  \begin{enumerate}
  \def\labelenumi{\alph{enumi})}
  \setcounter{enumi}{1}
  \tightlist
  \item
    In terms of error rate, how well do you think the tree associated
    with line 23 will perform on different data from the sample
    population. I could imagine there being a large error rate maybe
    somewhere above 50\%, because it's overfitting too much to the
    original data it will likely fail on new data.
  \end{enumerate}
\item
  \begin{enumerate}
  \def\labelenumi{\alph{enumi})}
  \setcounter{enumi}{2}
  \tightlist
  \item
    Consider the tree that makes no splits --- i.e.~the one that simply
    classifies according to the most likely class. How well will this
    tree classify new data from the same population. The root node error
    is 0.49161, which means if the tree simply classifies according to
    the most likely class, it would achieve an accuracy of around 50\%
    on new data from the same population, this is not very good. The
    tree will not classify new data from the same population well.
  \end{enumerate}
\item
  \begin{enumerate}
  \def\labelenumi{\alph{enumi})}
  \setcounter{enumi}{3}
  \tightlist
  \item
    Judging from the table, what appears to be your best choice of
    complexity parameter α? In what sense is your α value best? α =
    0.00517799, because this is the last split where the xerror changes
    much. Therefore it is the best value to choose.
  \end{enumerate}
\end{itemize}

\#Problem 5 - a) Write R code to fill in the matrix z to be as described
in the problem.

\begin{Shaded}
\begin{Highlighting}[]
\NormalTok{x }\OtherTok{=} \FunctionTok{matrix}\NormalTok{(}\FunctionTok{c}\NormalTok{(.}\DecValTok{7}\NormalTok{, .}\DecValTok{6}\NormalTok{, .}\DecValTok{5}\NormalTok{, .}\DecValTok{5}\NormalTok{, .}\DecValTok{5}\NormalTok{, .}\DecValTok{5}\NormalTok{, .}\DecValTok{7}\NormalTok{, .}\DecValTok{8}\NormalTok{, .}\DecValTok{7}\NormalTok{, .}\DecValTok{6}\NormalTok{, .}\DecValTok{7}\NormalTok{, .}\DecValTok{9}\NormalTok{, .}\DecValTok{8}\NormalTok{, .}\DecValTok{9}\NormalTok{ ),}\AttributeTok{byrow=}\NormalTok{T,}\AttributeTok{nrow=}\DecValTok{2}\NormalTok{);}
\NormalTok{z }\OtherTok{=} \FunctionTok{array}\NormalTok{(x,}\FunctionTok{c}\NormalTok{(}\DecValTok{2}\NormalTok{,}\DecValTok{7}\NormalTok{,}\DecValTok{2}\NormalTok{));}
\end{Highlighting}
\end{Shaded}

\begin{itemize}
\item
  \begin{enumerate}
  \def\labelenumi{\alph{enumi})}
  \setcounter{enumi}{1}
  \tightlist
  \item
    Using your z matrix create an R function that receives a vector of 7
    test answers which are either wrong or right. For instance, if the
    answers are c(0,0,0,0,1,1,1), that would mean the student answered
    only the last three questions correctly. The function should return
    the probability that the student has understood the subject, using a
    Naive Bayes classifier.
  \end{enumerate}
\end{itemize}

\begin{Shaded}
\begin{Highlighting}[]
\NormalTok{x }\OtherTok{=} \FunctionTok{matrix}\NormalTok{(}\FunctionTok{c}\NormalTok{(.}\DecValTok{7}\NormalTok{, .}\DecValTok{6}\NormalTok{, .}\DecValTok{5}\NormalTok{, .}\DecValTok{5}\NormalTok{, .}\DecValTok{5}\NormalTok{, .}\DecValTok{5}\NormalTok{, .}\DecValTok{7}\NormalTok{, .}\DecValTok{8}\NormalTok{, .}\DecValTok{7}\NormalTok{, .}\DecValTok{6}\NormalTok{, .}\DecValTok{7}\NormalTok{, .}\DecValTok{9}\NormalTok{, .}\DecValTok{8}\NormalTok{, .}\DecValTok{9}\NormalTok{ ),}\AttributeTok{byrow=}\NormalTok{T,}\AttributeTok{nrow=}\DecValTok{2}\NormalTok{);}
\NormalTok{z }\OtherTok{=} \FunctionTok{array}\NormalTok{(x,}\FunctionTok{c}\NormalTok{(}\DecValTok{2}\NormalTok{,}\DecValTok{7}\NormalTok{,}\DecValTok{2}\NormalTok{));}
\NormalTok{naive\_bayes }\OtherTok{=} \ControlFlowTok{function}\NormalTok{(answers)\{}
\NormalTok{  prior\_understand }\OtherTok{=} \FloatTok{0.2}
\NormalTok{  prior\_not }\OtherTok{=} \FloatTok{0.8} 
  
\NormalTok{  probs\_correct }\OtherTok{=}\NormalTok{ z[,,}\DecValTok{2}\NormalTok{] }\SpecialCharTok{*}\NormalTok{ answers }
\NormalTok{  probs\_incorrect }\OtherTok{=}\NormalTok{ z[,,}\DecValTok{1}\NormalTok{] }\SpecialCharTok{*}\NormalTok{ answers}
  
\NormalTok{  understand }\OtherTok{=}\NormalTok{ prior\_understand }\SpecialCharTok{*}\NormalTok{ probs\_correct }\SpecialCharTok{/}\NormalTok{ (probs\_correct }\SpecialCharTok{+}\NormalTok{ probs\_incorrect)}
\NormalTok{  not }\OtherTok{=}\NormalTok{ prior\_not }\SpecialCharTok{*}\NormalTok{ probs\_incorrect }\SpecialCharTok{/}\NormalTok{ (probs\_correct }\SpecialCharTok{+}\NormalTok{ probs\_incorrect)}
  
  \CommentTok{\#understand = prior\_understand * probs\_correct / (sum(answers)/7)}
  \CommentTok{\#not = prior\_not * probs\_incorrect / ((7{-}sum(answers))/7)}
  \FunctionTok{return}\NormalTok{(understand)}
  
\NormalTok{\}}
\NormalTok{answers }\OtherTok{=} \FunctionTok{c}\NormalTok{(}\DecValTok{0}\NormalTok{,}\DecValTok{0}\NormalTok{,}\DecValTok{0}\NormalTok{,}\DecValTok{0}\NormalTok{,}\DecValTok{1}\NormalTok{,}\DecValTok{1}\NormalTok{,}\DecValTok{1}\NormalTok{)}
\FunctionTok{naive\_bayes}\NormalTok{(answers)}
\end{Highlighting}
\end{Shaded}

\begin{verbatim}
##      [,1] [,2] [,3] [,4] [,5] [,6] [,7]
## [1,]  NaN  NaN  0.1  0.1  NaN  NaN  0.1
## [2,]  NaN  NaN  0.1  NaN  NaN  0.1  0.1
\end{verbatim}

\end{document}
